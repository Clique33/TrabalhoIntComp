\documentclass{article}
\usepackage[utf8]{inputenc}
\usepackage{amsmath}

\title{Iterated Local Search no Problema da Clique Máxima}
\author{Gabriel Cardoso de Carvalho}
\date{}

\begin{document}
\maketitle

\textbf{Resumo:} Esse artigo mostra os resultados de uma implementação do Iterated Local Search no problema da Clique Máxima, comparando com os resultados de outras implementações desse mesmo método e de outros métodos como o genético.

\section{Introdução}

O problema de encontrar a Clique Máxima (CM) é extremamente conhecido e estudado, pois inúmeros problemas práticos de diversas áreas diferentes, como biologia computacional, economia e análise de redes sociais podem ser modelados como CM. 
Além disso, a sua versão de decisão foi um dos primeiros problemas a serem provados NP-Completos.\par 

Ele pode ser definido da seguinte maneira, seja o grafo $G=(V,E)$ onde $V = 1,2, ... , n$ é o conjunto de vértices e $E \subseteq V \times V$ é o conjunto de arestas, uma Clique $C \subseteq V $é tal que $\forall i,j \in C, (i,j) \in E$ 
, ou seja, todos os vértices em $C$ são adjacentes entre si. Ou ainda, $C$ é um subgrafo completo de $G$. O problema da clique máxima é o problema de encontrar a clique de cardinalidade máxima
d0 grafo $G$.\par



\section{ILS}
\subsection{Geração da Solução Inicial}
\subsection{Busca Local}
\subsection{Perturbação}
\subsection{Critério de Aceitação}
\section{Resultados Experimentais}
\section{Conclusão}


\begin{thebibliography}{9}

\bibitem{kopt}
  Katayama, Kengo, Masashi Sadamatsu, and Hiroyuki Narihisa. 
"Iterated k-opt local search for the maximum clique problem." 
\textit{Lecture Notes in Computer Science} 4446 (2007): 84.

\bibitem{review}
Wu, Qinghua, and Jin-Kao Hao. 
"A review on algorithms for maximum clique problems." 
\textit{European Journal of Operational Research} 242.3 (2015): 693-709.

\bibitem{}

\end{thebibliography}

\end{document}