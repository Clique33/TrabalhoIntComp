\documentclass{article}
\usepackage[utf8]{inputenc}
\usepackage{amsmath}
\usepackage{color}
\usepackage[]{algorithm2e}

\title{Iterated Local Search no Problema da Clique Máxima}
\author{Gabriel Cardoso de Carvalho}
\date{}

\begin{document}
\maketitle

\textbf{Resumo:} Esse artigo mostra os resultados de uma implementação do Iterated Local Search no problema da Clique Máxima, comparando com os resultados de outras implementações desse mesmo método e de outros métodos como o genético.

\section{Introdução}

O problema de encontrar a Clique Máxima (CM) é extremamente conhecido e estudado, pois inúmeros problemas práticos de diversas áreas diferentes, como biologia computacional, economia e análise de redes sociais podem ser modelados como CM. 
Além disso, a sua versão de decisão foi um dos primeiros problemas a serem provados NP-Completos.\par 

Ele pode ser definido da seguinte maneira, seja o grafo $G=(V,E)$ onde $V = 1,2, ... , n$ é o conjunto de vértices e $E \subseteq V \times V$ é o conjunto de arestas, uma Clique $C \subseteq V $é tal que $\forall i,j \in C, (i,j) \in E$ 
, ou seja, todos os vértices em $C$ são adjacentes entre si. Ou ainda, $C$ é um subgrafo completo de $G$. O problema da clique máxima é o problema de encontrar a clique de cardinalidade máxima do grafo $G$.\par

Diversas soluções foram propostas, tanto métodos exatos quanto heurísticas e metaheurísticas \cite{review,pardaloshand,DIMACS2}. As propostas no geral tendem a utilizar o sistema \textit{breach and bound} nos métodos exatos e heurísticas gulosas em buscas locais, preferindo vértices de maior grau. De maneira geral, os algoritmos gulosos partem de uma clique $C$ inicial que contém apenas um vértice e um conjunto $N_C$ de vértices $v \in V$ que são os vértices vizinhos à $C$, ou seja, $\forall u \in C, v$ é adjacente à $u$. Daí o algoritmo adiciona vértices de  $N_C$ em $C$, escolhendo sempre o vértice de $N_C$ que tem o maior grau no subgrafo $G(N_C)$, até que $C$ seja \textit{maximal}, ou seja, até que não exista uma clique $C'$ maior que $C$ tal que $C \subseteq C'$. Em outros trabalhos esse método é chamado de \textit{Busca Local 1-opt } \cite{KLS}.\par

A metaheurística implementada nesse artigo é a \textit{Iterated Local Search (ILS)}, que pode ser resumida como uma metaheurística que cria, de maneira iterativa, uma sequência de soluções geradas por uma heurística interna (ou busca local) \cite{handbook}. É esperado que as soluções providas pelo ILS sejam melhores do que uma simples repetição da heurística de maneira aleatória.

\textcolor{red}{Nesse artigo é proposta uma implementação do ILS focada na aleatoriedade, de modo a comparar seu desempenho com métodos gulosos, como a implementação do IKLS de \textit{Katayama} \cite{kopt}, que utiliza a \textit{Busca Local k-opt (KLS)} \cite{KLS} como busca local, que é uma generalização da busca local 1-opt, onde adiciona-se $k$ vértices à clique por vez, permitindo retirar vértices da clique para isso, de maneira dinâmica, ou seja, o $k$ não é fixo,  e uma perturbação baseada na \textit{menor conecividade por arestas (LEC-KLS)}, onde escolhe-se um vértice $v$ que não pertence à $C$, de maneira que $v$ seja adjacente à menor quantidade de vértices de $C$. Então, adiciona-se $v$  à $C$ e remove de $C$ todos os vértices que não são vizinhos à $v$.}\par

A seção 2 apresenta como foi feita a implementação do ILS, enquanto a seção 3 cobre toda a implementação e os resultados experimentais. A seção 4 Apresenta as conclusões e os trabalhos futuros.

\section{ILS}

\begin{algorithm}[H]
 \KwData{Grafo $G$, inteiro $n_{iter}$}
 $s_0$ = GeraSolucaoinicial($G$)\;
 $s*$ = BuscaLocal($s_0$)\;
 $k$ = 0\;
 \While{$k$ for menor que $n_{iter}$}{
  $s'$ = Perturbacao($s*$)\;
  $s*'$ = BuscaLocal($s'$)\;
  $s*$ = CriterioAceitacao($s*, s*'$)\;
 $k_{++}$\;
 }
 \caption{Estrutura do ILS}
\end{algorithm}


Esta implementação do ILS segue o padrão de Glover \cite{handbook} utilizando o algoritmo 1, e os detalhes de cada função são descritos nas subseções seguintes. 



\subsection{Geração da Solução Inicial}
\subsection{Busca Local}
\subsection{Perturbação}
\subsection{Critério de Aceitação}
\section{Resultados Experimentais}
\section{Conclusão}


\begin{thebibliography}{58}

\bibitem{kopt}
  Katayama, Kengo, Masashi Sadamatsu, and Hiroyuki Narihisa. 
"Iterated k-opt local search for the maximum clique problem." 
\textit{Lecture Notes in Computer Science} 4446 (2007): 84.

\bibitem{review}
Wu, Qinghua, and Jin-Kao Hao. 
"A review on algorithms for maximum clique problems." 
\textit{European Journal of Operational Research} 242.3 (2015): 693-709.

\bibitem{pardaloshand}
I.M. Bomze, M. Budinich, P.M. Pardalos, and M. Pelillo. The maximum clique
problem. In D.-Z. Du and P.M. Pardalos, editors, \textit{Handbook of Combinatorial
Optimization (suppl. Vol. A)}, pp. 1–74. Kluwer, 1999.

\bibitem{DIMACS2}
D.S. Johnson and M.A. Trick. \textit{Cliques, Coloring, and Satisfiability}. Second DIMACS
Implementation Challenge, DIMACS Series in Discrete Mathematics and
Theoretical Computer Science. American Mathematical Society, 1996.

\bibitem{KLS}
 K. Katayama, A. Hamamoto, and H. Narihisa. An effective local search for the
maximum clique problem. \textit{Information Processing Letters}, Vol. 95, No. 5, pp.
503–511, 2005.

\bibitem{handbook}
Glover, Fred W., and Gary A. Kochenberger, eds. \textit{Handbook of metaheuristics}. Vol. 57. Springer Science \& Business Media, 2006.

\end{thebibliography}

\end{document}